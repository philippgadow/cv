\documentclass{article}

% Package definitions
\usepackage[
  a4paper,
  margin=1in,
  headsep=4pt, % separation between header rule and text
]{geometry}
\usepackage{xcolor}
\usepackage{fancyhdr}
\usepackage{amsmath}

% Document design choices
\textwidth6in
\setlength{\topmargin}{0in} \addtolength{\topmargin}{-\headheight}
\addtolength{\topmargin}{-\headsep}
\setlength{\oddsidemargin}{0in}

\oddsidemargin  0.0in \evensidemargin 0.0in \parindent0em
\pagestyle{fancy}\lhead{Research Statement} \rhead{Mar 2022}
\chead{{\large{\bf Philipp Gadow}}} \lfoot{} \rfoot{\bf \thepage} \cfoot{}


% Bibliography
\usepackage{biblatex}
\addbibresource{researchstatement_pgadow.bib}


\begin{document}
\raisebox{.5cm}

% Research topic in a nutshell
My research investigates new phenomena of physics beyond the Standard Model of particle physics (SM) and their possible connection with the nature of electroweak symmetry breaking.

The experiments studying collisions from the Large Hadron Collider (LHC) at the CERN laboratory in Geneva have confirmed the predictions of the SM. However, the SM is incomplete and likely to be an effective description of a more fundamental theory.
\bigskip

% Searches for heavy particles produced in association with top quarks
I am interested in \textbf{direct searches for new heavy particles} predicted by extensions of the SM by investigating rare and challengingly complex signatures that previous data analyses have not been able to resolve.
The large Yukawa coupling of the top quark to the Higgs boson suggests using it as a probe for new physics underlying the electroweak symmetry breaking.
While the current null results of \(t\overline{t}\) resonance searches provide stringent constraints on the mass of new particles at the TeV scale, depending on models, they have limited sensitivity to resonances coupling exclusively to third-generation-particles which require via top-associated production.

Currently, I am leading a search for heavy particles produced in association with top quarks targeting a four-top-quark final state based on the full ATLAS Run-2 \(pp\) dataset. The search explores the resonance mass range above \(1\,\)TeV in a model-independent way.
The complex final state requires an excellent reconstruction of the top quark candidates.
Substantial improvements in sensitivity can be achieved by exploiting methods of geometrical deep learning.
In the context of a DESY summer student project, I supervised studies on applying machine learning for the identification of top quark candidates. I plan to continue studies exploring the use of graph neural networks for the reconstruction of top quark decays in dense environments.

\medskip

% Flavour tagging
An essential component of searches involving top quarks is the \textbf{identification of jets containing \(b\)-hadrons}. More generally, \textbf{jet flavour tagging} underpins the results in many areas of the physics programme of the ATLAS experiment, such as the recent observations of the Higgs boson decay into bottom quarks.
In my role as co-convener of the ATLAS FTAG algorithms subgroup consisting of about 40 researchers, I am responsible for the development, study and maintenance of machine learning algorithms used by the collaboration.
Ideas from geometrical deep learning can also benefit the design of more powerful algorithms for jet flavour tagging. In particular, I want to study neural network architectures which provide improved performance for dense and boosted event topologies.

\medskip

% Dark matter searches
I am further interested in the possible connection of dark matter with electroweak symmetry breaking, exploring \textbf{signatures with missing transverse momentum} predicted by \textbf{dark matter models with extended Higgs sector}.
Searches for new physics at the LHC can probe signatures of yet unobserved dark matter particles which can only be detected by their recoil on visible particles, such as hadrons, electroweak bosons, and Higgs bosons.

I was the central analyser in teams of 15-30 physicists searching for dark matter production in association with hadronically decaying vector bosons, Higgs bosons decaying to b-quarks, and hypothetical dark Higgs bosons decaying to pairs of vector bosons. The searches pioneered novel analysis techniques for b-quark identification, background suppression, and boosted object reconstruction.
The results of the analysis targeting dark matter production in association with hadronically decaying vector bosons were additionally included in an extensive summary publication, for which I coordinated their integration and statistical interpretation in terms of a consistent model for dark matter production with an extended Higgs sector, and in a combination of ATLAS searches for invisible Higgs decays.
\medskip

% Reproducible research
Searches for new physics represent a significant investment of time and resources. In the light of a steadily increasing sensitivity to a growing number of BSM scenarios, a \textbf{powerful reinterpretation framework} enhances the impact of searches whose data is preserved. I coordinated and edited the first ATLAS public result using the RECAST reinterpretation framework to constrain a model predicting dark matter production in association with a dark Higgs boson decaying to b-quarks. I commit to the ideal of open and reproducible research for the benefit of the scientific community and am contributing to software trainings by the HEP Software Foundation.
\medskip

% Conclusion
These searches are enabled by the record precision on the dataset to be collected by the ATLAS experiment until 2025 and by improvements in experimental techniques. In conclusion, I am working on new searches exploiting the precision of the ATLAS Run-2 dataset and contributing towards successful operation of the detector during Run-3.

% Bibliography
% \printbibliography

\end{document}
