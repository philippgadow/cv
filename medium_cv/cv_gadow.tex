%!TEX encoding = UTF-8 Unicode
% CV LaTeX Template for CV/Resume
%
% This template has been downloaded and modified from:
% https://github.com/posquit0/Awesome-CV
%
% Author:
% Claud D. Park <posquit0.bj@gmail.com>
% http://www.posquit0.com
%
% Modified (March 2019) by:
% Jessica A. Rick <jrick@uwyo.edu>
% http://www.jessicarick.com
% 
%


%-------------------------------------------------------------------------------
% CONFIGURATIONS
%-------------------------------------------------------------------------------
% A4 paper size by default, use 'letterpaper' for US letter
\documentclass[11pt, letterpaper, draft]{academic-cv}

% Configure page margins with geometry
\geometry{left=0.75in, top=0.75in, right=0.75in, bottom=0.75in, footskip=.5cm}

% Specify the location of the included fonts
\fontdir[fonts/]

% If you would like to change the social information separator from a pipe (|) to something else
\renewcommand{\acvHeaderSocialSep}{\quad\textbar\quad}


%-------------------------------------------------------------------------------
%	PERSONAL INFORMATION
%	Comment any of the lines below if they are not required
%-------------------------------------------------------------------------------
\name{Philipp Gadow}{}
\position{Research Associate {\enskip\cdotp\enskip}University of Hamburg}
\address{Office 113 b, Building 62, Luruper Chaussee 149, 22761 Hamburg, Germany}

\orcid{0000-0003-4475-6734}
\email{philipp.gadow@cern.ch}
\homepage{http://me.pgadow.de}
% \github{philippgadow}
% \extrainfo{extra informations}


%-------------------------------------------------------------------------------
\begin{document}

% Print the header with above personal information
% Give optional argument to change alignment(C: center, L: left, R: right)
\makecvheader

% Print the footer with 3 arguments(<left>, <center>, <right>)
% Leave any of these blank if they are not needed
\makecvfooter
  {March 2025}
  {Philipp~·~Gadow ~~~ Curriculum Vitae}
  {\thepage}

\cvsection{Work history and education}
\begin{cventries}

  %---------------------------------------------------------
    \cventry
      {Research Staff} % Job title
      {University of Hamburg} % Institution
      {Hamburg, Germany} % Location
      {2024 -- Present} % Date(s)
      {
        \begin{cvitems} % Description(s) bullet points
          \item {Machine learning algorithms for jet flavour tagging, investigation of Higgs potential through searches for Higgs pair production, direct searches for dark matter with spherical proportional counters.}
          \item {Teaching, student supervision.}
        \end{cvitems}
      }

    \cventry
      {Senior Research Fellow} % Job title
      {CERN} % Institution
      {Meyrin, Switzerland} % Location
      {2023 -- 2024} % Date(s)
      {
        \begin{cvitems} % Description(s) bullet points
          \item {Searches for new phenomena in final states with top quarks, leadership of the ATLAS "Heavy Quarks, Top and Composite Higgs" physics sub-group, machine learning algorithms for jet flavour tagging and lepton isolation, silicon detector development for future collider experiments.}
          \item {Co-supervision of two undergraduate students and CERN summer students.}
        \end{cvitems}
      }
  
    \cventry
      {Quantum Universe Excellence Cluster Fellow} % Job title
      {Deutsches Elektronensynchrotron DESY} % Institution
      {Hamburg, Germany} % Location
      {2020 -- 2023} % Date(s)
      {
        \begin{cvitems} % Description(s) bullet points
          \item {Searches for heavy resonances with four-top-quark events, leadership of ATLAS "Flavour Tagging Algorithms" combined performance sub-group, reinterpretation of searches with active learning. }
          \item {Co-supervision of three PhD students and supervision of two summer students.}
        \end{cvitems}
      }
  
	  \cventry
	  {Dr. rer. nat. Physics} % Degree
	  {Technical University of Munich} % Institution
	  {Munich, Germany} % Location
	  {2016 -- 2020} % Date(s)
	  {
		\begin{cvitems} % Description(s) bullet points
		  \item {Advisor: PD Dr. Oliver Kortner.}
		  \item {Thesis: \href{https://cds.cern.ch/record/2744557}{Search for Dark Matter Produced in Association with Hadronically Decaying Bosons at \(\sqrt{s}\)~=~13\,TeV with the ATLAS Detector at the LHC}.}
		  \item {Tutor for experimental physics lectures and instructor for particle physics masterclasses in higher education schools.}
		\end{cvitems}
	  }
  
	\cventry
	  {Master of Science in Physics (Nuclear, Particle and Astrophysics)} % Degree
	  {Technical University of Munich} % Institution
	  {Munich, Germany} % Location
	  {2013 -- 2016} % Date(s)
	  {
		\begin{cvitems} % Description(s) bullet points
		  \item {Advisor: PD Dr. Oliver Kortner.}
		  \item {Thesis: \href{https://cds.cern.ch/record/2162398}{Development of a Concept for the Muon Trigger of the ATLAS Detector at the HL-LHC}.}
		  \item {Passed with distinction.}
		\end{cvitems}
	  }
  
	  \cventry
	  {Erasmus+ Student Mobility for Studies} % Degree
	  {University of Edinburgh} % Institution
	  {Edinburgh, United Kingdom} % Location
	  {Winter 13} % Date(s)
	  {
		\begin{cvitems} % Description(s) bullet points
		  \item {Courses (reported with ECTS, grade/mark): Philosophy of Science 1 (10 ECTS, B/64), Musical Acoustics (10 ECTS, A1/93), Quantum Physics (5 ECTS, A1/96), Quantum Theory (5 ECTS, A3/74), Relativistic Quantum Field Theory (5 ECTS, D/42).}
		\end{cvitems}
	  }
  
	  \cventry
	  {Teaching Degree for Upper Secondary Education “Gymnasiallehramt” (not completed)} % Degree
	  {Ludwig Maximilian University of Munich} % Institution
	  {Munich, Germany} % Location
	  {2012 -- 2016} % Date(s)
	  {
		\begin{cvitems} % Description(s) bullet points
		  \item {With subjects physics, mathematics, and educational science.}
		  \item {Completed educational science part (36/36 ECTS) and in­ten­si­ve in­ternship (duration of 1 year, com­bi­na­ti­on of SPS I and SPS II), partially completed mathematics part (66/105 ECTS) and physics part (84/105 ECTS), dropout in favour of PhD studies.}
		\end{cvitems}
	  }
  
	\cventry
	  {Bachelor of Science in Physics} % Degree
	  {Technical University of Munich} % Institution
	  {Munich, Germany} % Location
	  {2010 -- 2013} % Date(s)
	  {
		\begin{cvitems} % Description(s) bullet points
		  \item {Advisor: Prof. Dr. Laura Fabbietti.}
		  \item {Thesis: \href{https://www.das.ktas.ph.tum.de/DasDocs/Public/Bachelor_Theses/thesis_Gadow.pdf}{dE/dx studies with pion and electron tracks of the ALICE GEM IROC prototype}.}
		  \item {Passed with merit.}
		\end{cvitems}
	  }
  %---------------------------------------------------------
  \end{cventries}
  
\newpage

%-------------------------------------------------------------------------------
%	SECTION TITLE
%-------------------------------------------------------------------------------
\cvsection{Awards, Fellowships, \& Grants}


\begin{cvhonors}

%---------------------------------------------------------
  \cvhonor
    {ATLAS Outstanding Achievement Award 2024} % Award
    {for outstanding contributions to heavy flavour tagging algorithms based on
    Graph Neural Networks.} % Awarding Organization
    {} % Amount -- leave blank if not including amounts
    {2024} % Date(s)

%---------------------------------------------------------

  \cvhonor
    {Leadership Academy 8 Fellowship} % Award
    {German Scholars Organization.} % Awarding Organization
    {EUR 11,000} % Amount -- leave blank if not including amounts
    {2024} % Date(s)

%---------------------------------------------------------

    \cvhonor
    {Senior Research Fellowship} % Award
    {CERN.} % Awarding Organization
    {CHF 180,000} % Amount -- leave blank if not including amounts
    {2023 -- 2025} % Date(s)

%---------------------------------------------------------
  \cvhonor
    {Teaching Award "Goldene Kreide der Physikfachschaft"} % Award
    {Technical University of Munich.} % Awarding Organization
    {} % Amount
    {2014} % Date(s)

%---------------------------------------------------------
  \cvhonor
    {Full Scholarship} % Award
    {Studienstiftung des deutschen Volkes.} % Awarding Organization
    {} % Amount
    {2010 -- 2016} % Date(s)

\end{cvhonors}


\cvsection{Positions of Responsibility}

\begin{cvhonors}

%---------------------------------------------------------
  \cvhonor
    {ATLAS Physics Exotics HQT sub-group convener} % Name
    {group consisting of ~100 members with 2 conveners.} % Relationship
    {} % Location
    {2024} % Date(s)

%---------------------------------------------------------
  \cvhonor
    {ATLAS Control Room Shift Leader} % Name
    {overseeing data taking (over 50 shifts, of which in 2024: 23 shifts, in 2023: 24 shifts, and in 2022: 11 shifts).} % Relationship
    {} % Location
    {2022 -- 2024} % Date(s)

%---------------------------------------------------------
  \cvhonor
    {ATLAS Combined Performance Flavour Tagging Algorithms
    sub-group convener} % Name
    {group consisting of ~30 members with 2 conveners.} % Relationship
    {} % Location
    {2021 -- 2023} % Date(s)

%---------------------------------------------------------
  \cvhonor
    {ATLAS Physics Analysis Contact} % Name
    {for five analysis teams with teams consisting of between 5 to 40 members.} % Relationship
    {} % Location
    {2021 -- 2024} % Date(s)

%---------------------------------------------------------
\end{cvhonors}

%-------------------------------------------------------------------------------
%	SECTION TITLE
%-------------------------------------------------------------------------------
\cvsection{Mentoring}


%-------------------------------------------------------------------------------
%	CONTENT
%-------------------------------------------------------------------------------

\cvsubsection{(Co-)Supervision of Graduate Students}

\begin{cvhonors}

%---------------------------------------------------------
  \cvhonor
    {Dr. Alicia Wongel} % Name
    {PhD student co-supervised with Dr. Krisztian Peters.} % Relationship
    {University of Hamburg} % Location
    {2020 -- 2022} % Date(s)

%---------------------------------------------------------
  \cvhonor
    {Elisaveta Sitnikova} % Name
    {PhD student co-supervised with Dr. Krisztian Peters.} % Relationship
    {University of Hamburg} % Location
    {2022 -- 2023} % Date(s)

%---------------------------------------------------------
  \cvhonor
    {Jackson Barr} % Name
    {PhD student co-supervised with Dr. Krisztian Peters and Prof. Tim Scanlon.} % Relationship
    {UCL} % Location
    {2022 -- 2023} % Date(s)

%---------------------------------------------------------
\end{cvhonors}


\cvsubsection{(Co-)Supervision of Undergraduate Students}

\begin{cvhonors}
%---------------------------------------------------------
  \cvhonor
    {Milica Rajčić} % Name
    {CERN Summer Student Project, "Characterisation of the H2M monolithic pixel sensor ASIC".} % Relationship
    {University of Montenegro} % Location
    {2024} % Date(s)

%---------------------------------------------------------
  \cvhonor
    {Maya Kvaratskhelia} % Name
    {Boston Student Programme Project, "Improved lepton isolation for \(H(ZZ^{\ast})\) measurements".} % Relationship
    {Notre Dame University} % Location
    {2024} % Date(s)

%---------------------------------------------------------
  \cvhonor
    {Laura Winkler} % Name
    {Master Thesis Project, "Improved detection of charm jets using charged \(D^{\ast}\)-mesons".} % Relationship
    {University of Geneva} % Location
    {2023 -- 2024} % Date(s)

%---------------------------------------------------------
  \cvhonor
    {Stefan Katsarov} % Name
    {DESY Summer Student Project, "Jet flavour tagging project with training a deep-sets-based algorithm to identify \(b\)-jets".} % Relationship
    {University of Edinburgh} % Location
    {2022} % Date(s)

%---------------------------------------------------------
  \cvhonor
    {John Lawless} % Name
    {DESY Summer Student Project, "Machine learning techniques for top quark reconstruction in four-top-quark final states".} % Relationship
    {Iowa State University} % Location
    {2021} % Date(s)
%---------------------------------------------------------
\end{cvhonors}

%-------------------------------------------------------------------------------
%	SECTION TITLE
%-------------------------------------------------------------------------------
\cvsection{Teaching Experience}


%-------------------------------------------------------------------------------
%	CONTENT
%-------------------------------------------------------------------------------

\cvsubsection{Formal Pedagogical Training}

\begin{cventries}

  %---------------------------------------------------------
      \cventry
      {Educational science and Psychology for Upper Secondary Education Teaching Degree   “Gymnasiallehramt”} % Degree
      {Ludwig Maximilian University of Munich} % Institution
      {Munich, Germany} % Location
      {2012 -- 2016} % Date(s)
      {
        \begin{cvitems} % Description(s) bullet points
          \item {Formal education in pedagogics through completed educational science and psychology part (36/36 ECTS) and in­ten­si­ve in­ternship at Gymnasium Neufahrn (duration of 1 year, com­bi­na­ti­on of SPS I and SPS II).}
        \end{cvitems}
      }
  
    \cventry
      {Undergraduate tutor qualification certificate} % Degree
      {Technical University of Munich} % Institution
      {Munich, Germany} % Location
      {2011 -- 2012} % Date(s)
      {
        \begin{cvitems} % Description(s) bullet points
          \item {Provided by the central scientific
          institution for Higher Education Teaching "ProLehre (TUM)" of the Technical University of Munich.}
          \item{61 work units, corresponding to a workload of 45 hours consisting of a
          basic training course for undergraduate tutors, a reflexion session, a conflict training and an open training course, as well as specialised workshops on exercise design, utilisation of blackboards and presentation training. The training was complemented by two teaching consultations.}
        \end{cvitems}
      }
  %---------------------------------------------------------
  \end{cventries}
  
  

\cvsubsection{University Teaching}

\begin{cvhonors}

%---------------------------------------------------------
  \cvhonor
    {Physikalisches Praktikum I für Studierende der Naturwissenschaften} % Course
    {Lab Course.} % Course
    {UHH} % Location
    {2025} % Date(s)

%---------------------------------------------------------
  \cvhonor
    {Experimental Physics 2: Electromagnetism and Special Relativity} % Course
    {Tutorial.} % Course
    {TUM} % Location
    {2017} % Date(s)

%---------------------------------------------------------
  \cvhonor
    {Experimental Physics 1: Mechanics} % Course
    {Tutorial.} % Position
    {TUM} % Location
    {2016/17} % Date(s)

%---------------------------------------------------------
  \cvhonor
    {Tutorial: Experimental Physics 3: Optics} % Course
    {Tutorial.} % Position
    {TUM} % Location
    {2015/16} % Date(s)

%---------------------------------------------------------
  \cvhonor
    {Experimental Physics 2: Electromagnetism and Special Relativity} % Course
    {Tutorial.} % Committee
    {TUM} % Location
    {2014} % Date(s)

%---------------------------------------------------------
  \cvhonor
    {Experimental Physics 3: Optics} % Course
    {Tutorial.} % Committee
    {TUM} % Location
    {2013/14} % Date(s)

%---------------------------------------------------------
  \cvhonor
    {Experimental Physics 2: Electromagnetism and Special Relativity} % Course
    {Tutorial.} % Committee
    {TUM} % Location
    {2012} % Date(s)
%---------------------------------------------------------
  \cvhonor
    {Mathematics for physicists 1: Linear Algebra} % Course
    {Tutorial.} % Committee
    {TUM} % Location
    {2012/13} % Date(s)

%---------------------------------------------------------
  \cvhonor
    {Maths Introductory Course} % Course
    {for first-year students (three weeks).} % Committee
    {TUM} % Location
    {2012} % Date(s)

%---------------------------------------------------------
  \cvhonor
    {Mathematics for physicists 1: Linear Algebra} % Course
    {Tutorial.} % Committee
    {TUM} % Location
    {2011/12} % Date(s)

%---------------------------------------------------------
  \cvhonor
    {Maths Introductory Course} % Course
    {for first-year students (three weeks).} % Committee
    {TUM} % Location
    {2011} % Date(s)
%---------------------------------------------------------
\end{cvhonors}


\cvsubsection{Other Teaching}
\begin{cvhonors}
%---------------------------------------------------------
    \cvhonor
      {ATLAS Flavour Tagging Group Tutorials} % Course
      {Designed and supervised seven software tutorials for the ATLAS flavour tagging group. \href{Weblink}{https://ftag.docs.cern.ch/software/tutorials/}.} % Position
      {ATLAS} % Location
      {2021 -- 2023} % Date(s)
  %---------------------------------------------------------
    \cvhonor
      {ATLAS SUSY+HDBS+Exotics RECAST tutorial} % Course
      {Mentor for ATLAS virtual tutorial. \href{Indico agenda}{https://indico.cern.ch/event/1009271/}.} % Position
      {ATLAS} % Location
      {2021} % Date(s)
  %---------------------------------------------------------
    \cvhonor
      {CI/CD pipeline tutorial} % Course
      {Mentor for High Energy Physics Software Foundation virtual tutorial. \href{Indico agenda}{https://indico.cern.ch/event/904759/}.} % Position
      {HSF} % Location
      {2020} % Date(s)
  %---------------------------------------------------------
    \cvhonor
      {Docker training tutorial} % Course
      {Mentor for High Energy Physics Software Foundation virtual tutorial. \href{Indico agenda}{https://indico.cern.ch/event/934651/}.} % Position
      {HSF} % Location
      {2020} % Date(s)

  %---------------------------------------------------------
    \cvhonor
      {Secondary Education Teaching} % Course
      {"Lehr:werkstatt" intensive internship. Over the course of a year, I taught more than 250 hours in collaboration with an experienced teacher, independently prepared lessons, attended meetings and participated in school events.} % Position
      {Neufahrn Gymnasium, Germany} % Location
      {2014 -- 2015} % Date(s)
  %---------------------------------------------------------
\end{cvhonors}



\cvsection{Science Communication}

\begin{cvpubs}
  \cvpub{\textbf{Public talks}}
\end{cvpubs}

\begin{cvhonors}
  %---------------------------------------------------------
    \cvhonor
      {Keynote Talk for Higgs Boson Discovery 10th Anniversity} % Event/Organization
      {"Inspired by Higgs" at DESY / University of Hamburg. \href{Link to recording}{https://www.youtube.com/watch?v=sNs97If8vdw}.} % Position
      {} % Location
      {04.07.2022} % Date(s)
      
  %---------------------------------------------------------
    \cvhonor
      {Symposium on Science Communication} % Event/Organization
      {"Science Slams as a method of science communication" at Hamburg Research Academy.} % Position
      {} % Location
      {28.04.2022} % Date(s)

\end{cvhonors}

\begin{cvpubs}
  \cvpub{\textbf{Science Slams} are competetive events in which scientists present their research in a given time frame to a diverse audience in an entertaining way. I participated in over 30 such events with a \href{https://www.youtube.com/watch?v=ZBDvvXhFoZg}{talk about dark matter searches at the Large Hadron Collider}, including the Southern German championship. In 2022, I organised a two-day science communication workshop about science communication for doctoral researchers at DESY.}

  \cvpub{\textbf{CERN Outreach:} I was involved in several outreach programmes during my time as a CERN fellow. These include science shows, theatre-style presentations delivered by CERN scientists that promote science with interactive experiments, lasting 50 min
  with up to 200 visitors. I have presented 9 science shows during my time at CERN. I was also active as a CERN Guide and ATLAS Underground Guide. I have guided 24 private visits and 5 group visits. I was also involved in ATLAS virtual visits, live video connections where a CERN
  scientist guides a group through the ATLAS experiment. I have guided 7 virtual visits during my time at CERN.}

  \cvpub{\textbf{"3 minute thesis" competition judge} in panel for Hamburg Research Academy on 28.07.2022.}

  \cvpub{\textbf{ATLAS Masterclasses:} The ATLAS Masterclass programme is an educational outreach initiative that enables secondary
  school students to engage directly with particle physics. Students learn fundamental research
  methods by analysing measurements from actual collision events recorded by the ATLAS
  experiment.}

  \cvpub{\textbf{Netzwerk Teilchenwelt Teacher Training Programme:} I have imparted
  practical knowledge on cloud chamber experiments during teach-the-teacher training sessions for
  Bavarian teachers in the Bavarian state's teacher training centre in Gars am Inn.}

\end{cvpubs}

%-------------------------------------------------------------------------------
%	SECTION TITLE
%-------------------------------------------------------------------------------
\cvsection{Presentations}

%-------------------------------------------------------------------------------
%	CONTENT
%-------------------------------------------------------------------------------

%-------------------------------------------------------------------------------
\cvsubsection{Invited Talks}
%-------------------------------------------------------------------------------

\begin{cvpubs}
    \cvpub{July 2023. \textit{Heavy flavor jet tagging algorithms in ATLAS}. Invited talk: To b or not to b - CMS BTV Workshop 2023, Brussels, Belgium.\\\href{https://indico.cern.ch/event/1274182/contributions/5458302/}{https://indico.cern.ch/event/1274182/contributions/5458302/}.}
    
    \cvpub{Feb 2020. \textit{Dark matter searches with the ATLAS detector at the LHC}. Seminar talk: Cavendish Laboratory HEP Seminar, Cambridge, United Kingdom.}
\end{cvpubs}

%-------------------------------------------------------------------------------
\cvsubsection{International Conferences}
%-------------------------------------------------------------------------------

\begin{cvpubs}
    \cvpub{Mar 2024. \textit{A Scalable Platform for Training and Inference Using Kubeflow at CERN}. Workshop talk: Kubeflow Summit Europe, Paris, France.\\\href{https://sched.co/1YFhA/}{https://sched.co/1YFhA/}.}

    \cvpub{Feb 2024. \textit{Educational Outreach with AI-Assisted CERN Open Data Analysis}. Workshop talk: 1st Large Language Models in Physics Symposium, Hamburg, Germany.\\\href{https://indico.desy.de/event/38849/contributions/162122/}{https://indico.desy.de/event/38849/contributions/162122}.}

    \cvpub{Jul 2023. \textit{Searches for new phenomena in final states with 3rd generation quarks using the ATLAS detector}. Conference talk: SUSY2023, Southampton, United Kingdom.\\\href{https://indico.cern.ch/event/1214022/contributions/5461065/}{https://indico.cern.ch/event/1214022/contributions/5461065/}.}

    \cvpub{Jul 2022. \textit{Searches for new phenomena in final states with 3rd generation quarks using the ATLAS detector}. Conference talk: PHENO2023, Pittsburgh, United States of America.\\\href{https://indico.cern.ch/event/1089132/contributions/4855516/}{https://indico.cern.ch/event/1089132/contributions/4855516/}.}

    \cvpub{Jul 2019. \textit{ATLAS Highlights on Dark Matter Searches in Exotic Models}. Conference talk: XIII International Workshop on Interconnections between Particle Physics and Cosmology, Cartagena, Columbia.}

    \cvpub{Oct 2018. \textit{Search for dark matter produced in association with a Higgs boson decaying to bb}. Young Scientist Forum talk: Puzzle of Dark Matter Workshop, DESY Hamburg, Germany.\\\href{https://indico.desy.de/event/19155/contributions/34313/}{https://indico.desy.de/event/19155/contributions/34313/}.}

    \cvpub{Jun 2018. \textit{Search for Dark Matter in association with a hadronically decaying Z' vector boson with the ATLAS detector in pp collisions at 13 TeV}. Poster: Sixth Annual Conference on Large Hadron Collider Physics, Bologna, Italy.\\\href{https://indico.cern.ch/event/681549/contributions/2956249/}{https://indico.cern.ch/event/681549/contributions/2956249/}.}
\end{cvpubs}

%-------------------------------------------------------------------------------
\cvsubsection{National Conferences}
%-------------------------------------------------------------------------------

\begin{cvpubs}
    \cvpub{Sep 2019. \textit{Signal reweighting using BDTs}. Parallel talk: ATLAS Germany Meeting, Munich, Germany.\\\href{https://indico.cern.ch/event/811522/contributions/3541796}{https://indico.cern.ch/event/811522/contributions/3541796}.}

    \cvpub{Mar 2019. \textit{Dark Matter + Mono-h(bb): How to get rid of the multijet background using the object-based \(E_{\textrm{T}}^{\textrm{miss}}\) significance}. Parallel talk: DPG spring meeting, Aachen, Germany.}

    \cvpub{Sep 2018. \textit{Object-based \(E_{\textrm{T}}^{\textrm{miss}}\) significance in Mono-H(\(\overline{b}b\))}. Parallel talk: ATLAS Germany Meeting, Freiburg, Germany.\\\href{https://indico.cern.ch/event/700593/contributions/3092043/}{https://indico.cern.ch/event/700593/contributions/3092043/}.}

    \cvpub{Mar 2018. \textit{Search for Dark Matter produced in association with a hadronically decaying W or Z boson with ATLAS Run-2 data}. Parallel talk: DPG spring meeting, Würzburg, Germany.}

    \cvpub{Mar 2017. \textit{Search for Dark Matter produced in association with a hadronically decaying W or Z boson with ATLAS Run-2 data}. Parallel talk: DPG spring meeting, Münster, Germany.}

    \cvpub{Mar 2017. \textit{Development of a new Level-0 Muon Trigger for the ATLAS Experiment at High-Luminosity-LHC}. Parallel talk: DPG spring meeting, Münster, Germany.}

    \cvpub{Mar 2016. \textit{Development of fast track reconstruction algorithms for the ATLAS MDT-precision-chamber-based Level-0 Muon Trigger at HL-LHC}. Parallel talk: DPG spring meeting, Hamburg, Germany.}

    \cvpub{Mar 2016. \textit{Study of the MDT-precision-chamber-based Level-0 Muon Trigger selectivity for the ATLAS experiment at HL-LHC}. Parallel talk: DPG spring meeting, Hamburg, Germany.}
\end{cvpubs}

%-------------------------------------------------------------------------------
%	SECTION TITLE
%-------------------------------------------------------------------------------
\cvsection{Publications}

%-------------------------------------------------------------------------------
%	CONTENT
%-------------------------------------------------------------------------------

\textit{As a member of the ATLAS Collaboration, I am a co-author of over 500 peer-reviewed journal articles. I included those where I have made a substantial contribution below. The full list is available here through \href{https://inspirehep.net/}{https://inspirehep.net}, which also serves as the source for the number of citations of the publications (updated on 28.05.2024).}

%--------------------------------------------------------------
\cvsubsection{Selection of five most significant publications}
%--------------------------------------------------------------

\begin{cvpubs}
    \cvpub{1. \textbf{ATLAS Collaboration}. 2024. "Search for heavy resonances in four-top-quark final states in \(pp\) collisions at \(\sqrt{s} = 13\,\)TeV with the ATLAS detector." Eur. Phys. J. C 84 (2024) 157.\\\href{https://doi.org/10.1140/epjc/s10052-023-12318-9}{https://doi.org/10.1140/epjc/s10052-023-12318-9}. 4 citations.\\
    \textit{Personal contributions:} leadership of analysis, statistical analysis, signal simulation, analysis software.}

    \cvpub{2. \textbf{ATLAS Collaboration}. 2023. "ATLAS flavour-tagging algorithms for the LHC Run 2 \(pp\) collision dataset." Eur.Phys.J.C 83 (2023) 7, 681.\\\href{https://doi.org/10.1140/epjc/s10052-023-11699-1}{https://doi.org/10.1140/epjc/s10052-023-11699-1}. 160 citations.\\\textit{Personal contributions:} leadership of the publication, performance studies.}

    \cvpub{3. \textbf{ATLAS Collaboration}. 2021. "Search for dark matter produced in association with a Standard Model Higgs boson decaying into \(b\)-quarks using the full Run 2 dataset from the ATLAS detector." JHEP 11 (2021) 209.\\\href{https://doi.org/10.1007/JHEP11(2021)209}{https://doi.org/10.1007/JHEP11(2021)209}. 61 citations.\\\textit{Personal contributions:} analysis software maintenance, derivation software and requests, optimisation of missing transverse momentum significance, study of tight-jet cleaning.}

    \cvpub{4. \textbf{ATLAS Collaboration}. 2021. "Search for dark matter produced in association with a dark Higgs boson decaying to \(WW\) or \(ZZ\) in fully hadronic final states using \(pp\) collisions at \(\sqrt{s} = 13\,\)TeV recorded with the ATLAS detector." Phys. Rev. Lett. 126, 121802.\\\href{https://doi.org/10.1103/PhysRevLett.126.121802}{https://doi.org/10.1103/PhysRevLett.126.121802}. 23 citations.\\\textit{Personal contributions:} main analyser (of 2), analysis software development and maintenance, analysis strategy, track-assisted-reclustered jet optimisation, statistical analysis, signal simulation.}

    \cvpub{5. \textbf{ATLAS Collaboration}. 2018. "Search for dark matter in events with a hadronically decaying vector boson and missing transverse momentum in \(pp\) collisions at \(\sqrt{s} = 13\,\)TeV with the ATLAS detector." JHEP 10 (2018) 180.\\\href{https://doi.org/10.1007/JHEP10(2018)180}{https://doi.org/10.1007/JHEP10(2018)180}. 132 citations.\\\textit{Personal contributions:} main analyser (of 2), analysis software maintenance, estimate of multi-jet background, statistical analysis.}
\end{cvpubs}

%---------------------------------------------------------
\cvsubsection{Published peer-reviewed journal articles}
%---------------------------------------------------------

\begin{cvpubs}
    \cvpub{1. \textbf{Barr, S., P. Gadow., et al.} 2024. "Umami: A Python toolkit for jet flavour tagging." Journal of Open Source Software, 9(102), 5833.\\\href{https://doi.org/10.21105/joss.05833}{https://doi.org/10.21105/joss.05833}. 0 citations.}

    \cvpub{2. \textbf{ATLAS Collaboration}. 2024. "Search for top-philic heavy resonances in \(pp\) collisions at \(\sqrt{s} = 13\,\)TeV with the ATLAS detector." Eur. Phys. J. C 84 (2024) 157.\\\href{https://doi.org/10.1140/epjc/s10052-023-12318-9}{https://doi.org/10.1140/epjc/s10052-023-12318-9}. 4 citations.}

    \cvpub{3. \textbf{ATLAS Collaboration}. 2023. "Search for single vector-like \(B\) quark production and decay via \(B \rightarrow bH(bb)\) in \(pp\) collisions at \(\sqrt{s} = 13\,\)TeV with the ATLAS detector." JHEP 11 (2023) 168.\\\href{https://doi.org/10.1007/JHEP11(2023)168}{https://doi.org/10.1007/JHEP11(2023)168}. 7 citations.}

    \cvpub{4. \textbf{ATLAS Collaboration}. 2023. "ATLAS flavour-tagging algorithms for the LHC Run 2 \(pp\) collision dataset." Eur.Phys.J.C 83 (2023) 7, 681.\\\href{https://doi.org/10.1140/epjc/s10052-023-11699-1}{https://doi.org/10.1140/epjc/s10052-023-11699-1}. 160 citations.}

    \cvpub{5. \textbf{ATLAS Collaboration}. 2021. "Search for dark matter produced in association with a Standard Model Higgs boson decaying into \(b\)-quarks using the full Run 2 dataset from the ATLAS detector." JHEP 11 (2021) 209.\\\href{https://doi.org/10.1007/JHEP11(2021)209}{https://doi.org/10.1007/JHEP11(2021)209}. 61 citations.}

    \cvpub{6. \textbf{ATLAS Collaboration}. 2021. "Search for dark matter produced in association with a dark Higgs boson decaying to \(WW\) or \(ZZ\) in fully hadronic final states using \(pp\) collisions at \(\sqrt{s} = 13\,\)TeV recorded with the ATLAS detector." Phys. Rev. Lett. 126, 121802.\\\href{https://doi.org/10.1103/PhysRevLett.126.121802}{https://doi.org/10.1103/PhysRevLett.126.121802}. 23 citations.}

    \cvpub{7. \textbf{ATLAS Collaboration}. 2019. "Combination of Searches for Invisible Higgs Boson Decays with the ATLAS Experiment." Phys. Rev. Lett. 122, 231801.\\\href{https://doi.org/10.1103/PhysRevLett.122.231801}{https://doi.org/10.1103/PhysRevLett.122.231801}. 197 citations.}

    \cvpub{8. \textbf{ATLAS Collaboration}. 2019. "Constraints on mediator-based dark matter and scalar dark energy models using \(\sqrt{s} = 13\,\)TeV \(pp\) collision data collected by the ATLAS detector." JHEP 1905 (2019) 142.\\\href{https://doi.org/10.1007/JHEP05(2019)142}{https://doi.org/10.1007/JHEP05(2019)142}. 165 citations.}

    \cvpub{9. \textbf{ATLAS Collaboration}. 2018. "Search for dark matter in events with a hadronically decaying vector boson and missing transverse momentum in \(pp\) collisions at \(\sqrt{s} = 13\,\)TeV with the ATLAS detector." JHEP 10 (2018) 180.\\\href{https://doi.org/10.1007/JHEP10(2018)180}{https://doi.org/10.1007/JHEP10(2018)180}. 132 citations.}
\end{cvpubs}

%---------------------------------------------------------
\cvsubsection{Conference Publications}
%---------------------------------------------------------

    \textit{The ATLAS Collaboration also produces internally-reviewed “conference notes” and "public notes" in advance of conferences, so that results may be discussed with colleagues in other experiments and in the theory community. Conference notes are removed from the following list when they have been superseeded by a peer-reviewed publication.}

\begin{cvpubs}
    \cvpub{1. \textbf{Bhatti, Z.}, K. Cranmer, I. Espejo, L. Heinrich, P. Gadow, P. Rieck, J. von Ahnen. "Efficient Search for New Physics Using Active Learning in the ATLAS Experiment". 2024. EPJ Web Conf. 295 (2024) 09013.\\\href{https://doi.org/10.1051/epjconf/202429509013}{https://doi.org/10.1051/epjconf/202429509013}.}

    \cvpub{2. \textbf{ATLAS Collaboration}. "Active Learning reinterpretation of an ATLAS Dark Matter search constraining a model of a dark Higgs boson decaying to two b-quarks." 2022. ATL-PHYS-PUB-2022-045.\\\href{http://cds.cern.ch/record/2839789}{http://cds.cern.ch/record/2839789}.}

    \cvpub{3. \textbf{Malik, S.}, P. Gadow et al. "Software Training in HEP." 2021. Comput.Softw.Big Sci. 5 (2021) 1, 22.\\\href{https://doi.org/10.1007/s41781-021-00069-9}{https://doi.org/10.1007/s41781-021-00069-9}.}

    \cvpub{4. \textbf{Cieri, D.}, P. Gadow et al. "A Lightweight First-Level Muon Track Trigger for Future Hadron Collider Experiments." 2019. PoS TWEPP2018 (2019) 051.\\\href{https://doi.org/10.22323/1.343.0051}{https://doi.org/10.22323/1.343.0051}.}

    \cvpub{5. \textbf{Abovyan, S.}, P. Gadow et al. "First-level muon track trigger for future hadron collider experiments." 2019. Nucl.Instrum.Meth.A 936 (2019) 321-322.\\\href{https://doi.org/10.1016/j.nima.2019.01.035}{https://doi.org/10.1016/j.nima.2019.01.035}.}

    \cvpub{6. \textbf{ATLAS Collaboration}. 2019. "RECAST framework reinterpretation of an ATLAS Dark Matter Search constraining a model of a dark Higgs boson decaying to two \(b\)-quarks." ATL-PHYS-PUB-2019-032.\\\href{https://cds.cern.ch/record/2686290}{https://cds.cern.ch/record/2686290}.}

    \cvpub{7. \textbf{ATLAS collaboration}. 2018. "Search for Dark Matter Produced in Association with a Higgs Boson Decaying to bb at √s = 13 TeV with the ATLAS Detector using 79.8\(\,\)fb\(^{-1}\) of \(pp\) collisions." ATLAS-CONF-2018-039.\\\href{https://cds.cern.ch/record/2632344}{https://cds.cern.ch/record/2632344}.}

    \cvpub{8. \textbf{Gadow, P.} 2018. "Search for dark matter produced in association with a hadronically decaying \(Z'\) vector boson with the ATLAS detector at the LHC." PoS LHCP2018 (2018) 016.\\\href{https://doi.org/10.22323/1.321.0016}{https://doi.org/10.22323/1.321.0016}.}

    \cvpub{9. \textbf{Abovyan, S.}, P. Gadow et al. "Hardware Implementation of a Fast Algorithm for the Reconstruction of Muon Tracks in ATLAS Muon Drift-Tube Chambers for the First-Level Muon Trigger at the HL-LHC." 2017. Proceedings, 2017 IEEE Nuclear Science Symposium and Medical Imaging Conference and 24th international Symposium on Room-Temperature Semiconductor X-Ray \& Gamma-Ray Detectors (NSS/MIC 2017).\\\href{https://doi.org/10.1109/NSSMIC.2017.8532900}{https://doi.org/10.1109/NSSMIC.2017.8532900}.}

    \cvpub{10. \textbf{Gadow, P.}, O. Kortner, S. Kortner, H. Kroha, F. Müller, R. Richter. "Performance of a First-Level Muon Trigger with High Momentum Resolution Based on the ATLAS MDT Chambers for HL-LHC." 2016. Proceedings, 2015 IEEE Nuclear Science Symposium and Medical Imaging Conference (NSS/MIC 2015).\\\href{https://doi.org/10.1109/NSSMIC.2015.7581794}{https://doi.org/10.1109/NSSMIC.2015.7581794}.}

    \cvpub{11. \textbf{Nowak, S.}, P. Gadow et al. "Optimisation of the Read-out Electronics of Muon Drift-Tube Chambers for Very High Background Rates at HL-LHC and Future Colliders." 2016. Proceedings, 2015 IEEE Nuclear Science Symposium and Medical Imaging Conference (NSS/MIC 2015).\\\href{https://doi.org/10.1109/NSSMIC.2015.7581815}{https://doi.org/10.1109/NSSMIC.2015.7581815}.}

\end{cvpubs}

%---------------------------------------------------------
\cvsubsection{Technical Design Reports}
%---------------------------------------------------------

\begin{cvpubs}
    \cvpub{1. \textbf{ATLAS Collaboration}. 2017. "Technical Design Report for the Phase-II Upgrade of the ATLAS Muon Spectrometer." CERN-LHCC-2017-017.\\\href{https://cds.cern.ch/record/2285580}{https://cds.cern.ch/record/2285580}.}
\end{cvpubs}


%-------------------------------------------------------------------------------
\end{document}
