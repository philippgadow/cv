%% LaTeX CV file for Philipp Gadow
%% Copyright 2006-2015 Xavier Danaux (xdanaux@gmail.com).
%
% This work may be distributed and/or modified under the
% conditions of the LaTeX Project Public License version 1.3c,
% available at http://www.latex-project.org/lppl/.


\documentclass[11pt,a4paper,sans]{moderncv}        % possible options include font size ('10pt', '11pt' and '12pt'), paper size ('a4paper', 'letterpaper', 'a5paper', 'legalpaper', 'executivepaper' and 'landscape') and font family ('sans' and 'roman')

%% custom commands
\newcommand{\arxiv}[1]{\href{http://arxiv.org/abs/#1}{\texttt{arXiv:#1}}}

\newlength{\savedindent}

% moderncv themes
\moderncvstyle{classic}                             % style options are 'casual' (default), 'classic', 'banking', 'oldstyle' and 'fancy'
\moderncvcolor{grey}                               % color options 'black', 'blue' (default), 'burgundy', 'green', 'grey', 'orange', 'purple' and 'red'
%\renewcommand{\familydefault}{\sfdefault}         % to set the default font; use '\sfdefault' for the default sans serif font, '\rmdefault' for the default roman one, or any tex font name
%\nopagenumbers{}                                  % uncomment to suppress automatic page numbering for CVs longer than one page

% character encoding
%\usepackage[utf8]{inputenc}                       % if you are not using xelatex ou lualatex, replace by the encoding you are using
%\usepackage{CJKutf8}                              % if you need to use CJK to typeset your resume in Chinese, Japanese or Korean

% adjust the page margins
\usepackage[scale=0.75]{geometry}
%\setlength{\hintscolumnwidth}{3cm}                % if you want to change the width of the column with the dates
%\setlength{\makecvheadnamewidth}{10cm}            % for the 'classic' style, if you want to force the width allocated to your name and avoid line breaks. be careful though, the length is normally calculated to avoid any overlap with your personal info; use this at your own typographical risks...


\usepackage{bm}

% personal data
\name{Philipp}{Gadow}
\title{Curriculum Vitae}                               % optional, remove / comment the line if not wanted
\address{Große Rainstr. 87}{22765 Hamburg}{Germany}% optional, remove / comment the line if not wanted; the "postcode city" and "country" arguments can be omitted or provided empty
%% \phone[mobile]{+1~(234)~567~890}
% optional, remove / comment the line if not wanted; the optional "type" of the phone can be "mobile" (default), "fixed" or "fax"
%% \phone[fixed]{+2~(345)~678~901}
%% \phone[fax]{+3~(456)~789~012}
\email{philipp.gadow@mytum.de}                               % optional, remove / comment the line if not wanted
\homepage{me.pgadow.de}                         % optional, remove / comment the line if not wanted
%\social[linkedin]{john.doe}                        % optional, remove / comment the line if not wanted
%\social[xing]{john\_doe}                           % optional, remove / comment the line if not wanted
%\social[twitter]{jdoe}                             % optional, remove / comment the line if not wanted
% \social[github]{philippgadow}                     % optional, remove / comment the line if not wanted
%\social[gitlab]{https://gitlab.cern.ch/pgadow}     % optional, remove / comment the line if not wanted
%\social[skype]{jdoe}                               % optional, remove / comment the line if not wanted
%\extrainfo{additional information}                 % optional, remove / comment the line if not wanted
%\quote{Some quote}                                 % optional, remove / comment the line if not wanted

% bibliography adjustements (only useful if you make citations in your resume, or print a list of publications using BibTeX)
%   to show numerical labels in the bibliography (default is to show no labels)
%\makeatletter\renewcommand*{\bibliographyitemlabel}{\@biblabel{\arabic{enumiv}}}\makeatother
\renewcommand*{\bibliographyitemlabel}{[\arabic{enumiv}]}
%   to redefine the bibliography heading string ("Publications")
%\renewcommand{\refname}{Articles}

% bibliography with mutiple entries
%\usepackage{multibib}
%\newcites{book,misc}{{Books},{Others}}
%----------------------------------------------------------------------------------
%            content
%----------------------------------------------------------------------------------
\begin{document}
%\begin{CJK*}{UTF8}{gbsn}                          % to typeset your resume in Chinese using CJK
%-----       resume       ---------------------------------------------------------
\makecvtitle

\section{Employment and education}
\cventry{2020+}{Postdoctoral Research Fellow}{Deutsches Elektronensynchrotron DESY}{Hamburg, Germany}{}{}{}

\cventry{2016--2020}{PhD Physics}{Max-Planck-Institut für Physik}{Munich, Germany}{}{%
Advisor: PD Dr. Oliver Kortner \\%
\href{https://cds.cern.ch/record/2744557}{Thesis: Search for Dark Matter Produced in Association with Hadronically Decaying Bosons at \(\sqrt{s}\)~=~13\,TeV with the ATLAS Detector at the LHC} %
}
\cventry{2013--2016}{M.Sc. Particle Physics}{Technical University of Munich}{Munich, Germany}{}{%
\href{https://cds.cern.ch/record/2162398}{Thesis: Development of a Concept for the Muon Trigger of the ATLAS Detector at the HL-LHC}%
}
\cventry{Winter 2013}{Erasmus SMS}{University of Edinburgh}{Edinburgh, United Kingdom}{}{}
\cventry{2010--2013}{B.Sc. Physics}{Technical University of Munich}{Munich, Germany}{}{%
\href{https://www.das.ktas.ph.tum.de/DasDocs/Public/Bachelor_Theses/thesis_Gadow.pdf}{Thesis: dE/dx studies with pion and electron tracks of the ALICE GEM IROC prototype}%
}


\section{Leadership}
\cventry{2021+}{Flavour tagging algorithms sub-group convener}{ATLAS experiment}{}{}{Co-coordination (2 conveners) of a group consisting of 40 physicists working on machine learning algorithms for the identification of heavy-flavour jets at the ATLAS experiment.}
\cventry{2021+}{Analysis contact}{ATLAS experiment}{}{}{Co-coordination (2 contacts) of the heavy resonances search in four-top-quark final states.}


\section{Research experience}
\subsection{Searches for heavy particles}
\cventry{2020+}{Search for heavy resonances in four-top-quark final states}{}{}{}{
	\emph{Explicit resonance search in challenging four-top-quark final state.}\newline
	I co-lead the analysis team, maintained the analysis software and produced the inputs for the statistical analysis. Further, I contributed to fitting studies and simulated the signal process.
}
\subsection{Searches for dark matter}
\cventry{2018--2021}{Dark matter search with dark Higgs bosons decaying to vector boson pairs}{}{}{}{
	\emph{The first exploration of the signature with missing transverse momentum and resonantly produced vector boson pair, using a novel track-assisted-reclustering jet algorithm.}\newline
	I was the central analyser, responsible for all essential parts of this new analysis, involving signal process simulation, analysis strategy design, analysis software, distributed computing, and statistical analysis.
}
\cventry{2017--2021}{Dark matter searches with Higgs bosons decaying to $b$-quarks}{}{}{}{
	\emph{Unprecedented sensitivity to Higgs bosons with large recoil by using jets built from inner detector tracks with a variable radius. The results provide highly competitive constraints on dark matter models with extended Higgs sector.}\newline
	I was a central analyser in the search using the partial Run-2 data collected during 2015--2017 and contributed to the full Run-2 search. In the former, I was responsible for the analysis software, producing the fit inputs, the statistical analysis, the commissioning of a new object-based missing transverse momentum significance, and the multijet background estimate. In the latter, I contributed to maintaining the derivation datasets and the optimisation of the object-based missing transverse momentum significance-based selection requirements.
}
\cventry{2016--2018}{Dark matter search with hadronically decaying vector bosons}{}{}{}{
	\emph{Reconstruction of hadronic vector boson decays using jet substructure and jet flavour tagging information. Relevant signature for simplified dark matter models, models with extended Higgs sector and searches for invisible Higgs boson decays.}\newline
	I was a central analyser (of 2), responsible for the analysis software, distributed computing, and statistical analysis. I developed and evaluated a method for estimating the multijet background. Further, I was a liaison for incorporating the results for a summary publication, their reinterpretation in terms of a model with an extended Higgs sector and contributed to their use in a combination of searches for invisible Higgs boson decays.
}
\subsection{ATLAS combined performance}
\cventry{2021+}{Flavour tagging algorithms}{}{}{}{
	\emph{Jet flavour tagging underpins a large part of the ATLAS physics programme. Machine learning techniques are used for inferring the jet flavour.}\newline
	I contribute to the development of the next generation of recommended flavour tagging algorithms for Run-3 ATLAS physics analysis.
}
\subsection{Reinterpretation and reproducible research}
\cventry{2019+}{ATLAS RECAST reinterpretation framework}{}{}{}{
	\emph{Preservation and automated reinterpretation of searches drastically increase their relevance to a broader class of theoretical models. All recently published ATLAS searches are required to provide a RECAST reinterpretation framework implementation.}\newline
	I coordinated the reinterpretation of a dark matter search with Higgs bosons decaying to \(b\)-quarks in terms of a dark Higgs model with the RECAST framework and co-edited the first dedicated public note on RECAST. Further, I am interested in improving the limit setting in searches by exploiting active learning algorithms based on RECAST.
}
\subsection{Upgrade studies}
\cventry{2015--2017}{First-level muon trigger for High-Luminosity-LHC}{}{}{}{
	\emph{Highly selective first-level muon triggers are essential for the ATLAS experiment at the High-Luminosity LHC. Including the precision monitored-drift-tube information substantially increases the trigger selectivity.}\newline
	I studied a concept for a highly selective muon trigger, which is included in the Technical Design Report for the Phase-II Upgrade of the ATLAS Muon Spectrometer. Further, I studied fast track reconstruction algorithms which can be applied at the trigger-level.
}

\newpage

\section{Talks and posters}
\subsection{Invited talks}
\cventry{Feb 2020}{Dark matter searches with the ATLAS detector at the LHC}{Cavendish Laboratory HEP Seminar}{Cambridge, United Kingdom}{seminar talk}{}

\subsection{International conferences}
\cventry{Jul 2019}{ATLAS Highlights on Dark Matter Searches in Exotic Models}{XIII International Workshop on Interconnections between Particle Physics and Cosmology}{Cartagena, Columbia}{\href{https://fisindico.uniandes.edu.co/indico/contributionDisplay.py?contribId=5&sessionId=20&confId=78}{conference talk}}{}

\cventry{Oct 2018}{Search for dark matter produced in association with a Higgs boson decaying to bb}{Puzzle of Dark Matter Workshop}{DESY Hamburg, Germany}{\href{https://indico.desy.de/indico/event/19155/session/9/contribution/51/material/slides/0.pdf}{Young Scientist Forum talk}}{}

\cventry{Jun 2018}{Search for Dark Matter in association with a hadronically decaying Z' vector boson with the ATLAS detector in pp collisions at 13 TeV}{Sixth Annual Conference on Large Hadron Collider Physics}{Bologna, Italy}{\href{https://indico.cern.ch/event/681549/contributions/2956249/}{poster}}{}


% \subsection{National conferences}
% \cventry{Sep 2019}{Signal reweighting using BDTs}{ATLAS Germany Meeting}{Munich, Germany}{\href{https://indico.cern.ch/event/811522/contributions/3541796}{parallel talk}}{}

% \cventry{Mar 2019}{Dark Matter + Mono-h(bb): How to get rid of the multijet background using the object-based $E_{\textrm{T}}^{\textrm{miss}}$ significance}{DPG spring meeting}{Aachen, Germany}{parallel talk}{}

% \cventry{Sep 2018}{Object-based $E_{\textrm{T}}^{\textrm{miss}}$ significance in Mono-H($\overline{b}b$)}{ATLAS Germany Meeting}{Freiburg, Germany}{\href{https://indico.cern.ch/event/700593/contributions/3092043/}{parallel talk}}{}

% \cventry{Mar 2018}{Search for Dark Matter produced in association with a hadronically decaying W or Z boson with ATLAS Run-2 data}{DPG spring meeting}{Würzburg, Germany}{parallel talk}{}

% \cventry{Mar 2017}{Search for Dark Matter produced in association with a hadronically decaying W or Z boson with ATLAS Run-2 data}{DPG spring meeting}{Münster, Germany}{parallel talk}{}

% \cventry{Mar 2017}{Development of a new Level-0 Muon Trigger for the ATLAS Experiment at High-Luminosity-LHC}{DPG spring meeting}{Münster, Germany}{parallel talk}{}

% \cventry{Mar 2016}{Development of fast track reconstruction algorithms for the ATLAS MDT-precision-chamber-based Level-0 Muon Trigger at HL-LHC}{DPG spring meeting}{Hamburg, Germany}{parallel talk}{}

% \cventry{Mar 2016}{Study of the MDT-precision-chamber-based Level-0 Muon Trigger selectivity for the ATLAS experiment at HL-LHC}{DPG spring meeting}{Hamburg, Germany}{parallel talk}{}


\section{Awards}
\cventry{2010-2016}{Full scholarship}{}{}{}{%
	Studienstiftung des deutschen Volkes (German Academic Scholarship Foundation) \\%
	\textit{The German Academic Scholarship Foundation is Germany's largest and most prestigious scholarship foundation. Scholarships are awarded to fewer than 0.5\% of German students.}
}
\cventry{2014}{Teaching award}{}{}{}{Goldene Kreide der Physikfachschaft \\%
	\textit{The ``Goldene Kreide'' is awarded annually by the student council of the physics department to distinguish outstanding student tutors.}
}


\section{Schools}
\cventry{Jul 2019}{Fifth Machine Learning in High Energy Physics Summer School 2019
}{DESY}{Hamburg, Germany}{10 days}{}
\cventry{Sep 2017}{49. Herbstschule für Hochenergiephysik 2019}{University of Siegen}{Maria Laach, Germany}{10 days}{}


\section{Student supervision}
\subsection{Graduate Students}
\cventry{2022+}{Jackson Barr}{}{DESY / University College London (Physics)}{}{Flavour tagging algorithm development}
\cventry{2021+}{Elisaveta Sitnikova}{}{DESY / University of Hamburg (Physics)}{}{Search for heavy resonances in four-top-quark final states}
\cventry{2020--2022}{Alicia Wongel}{}{DESY / University of Hamburg (Physics)}{}{Search for heavy resonances in four-top-quark final states}
\cventry{2020--2022}{Janik von Ahnen}{}{DESY / University of Hamburg (Physics)}{}{Flavour tagging algorithm development, limits on a dark Higgs model with active learning}
\newpage
\subsection{Undergraduate Students}
\cventry{2021}{John Lawless}{}{DESY Summer Student}{}{Machine learning techniques for top quark reconstruction in four-top-quark final states}


\section{Teaching}
\cventry{2021}{SUSY+HDBS+Exotics RECAST Tutorial}{HEP Software Foundation}{}{mentor}{}
\cventry{2020}{Docker training}{HEP Software Foundation}{}{mentor}{}
\cventry{2020}{CI/CD pipelines training}{HEP Software Foundation}{}{mentor}{}
\cventry{2016/17}{Mechanics}{TU Munich}{}{tutor}{}
\cventry{2016/17}{Mechanics}{TU Munich}{}{tutor}{}
\cventry{2012, 2014, 2017}{Electromagnetism and Special Relativity}{TU Munich}{}{tutor}{}
\cventry{2012/13, 2015/16}{Optics}{TU Munich}{}{tutor}{}
\cventry{2011/12, 2012/13}{Linear Algebra}{TU Munich}{}{tutor}{}
\cventry{2011/12, 2012/13}{Maths introductory course for first-year students}{TU Munich}{}{tutor}{}


\section{Outreach}
\cventry{}{ATLAS Masterclass}{}{}{}{%
High school students learn about the fundamentals of particle physics in lectures and engage in hands-on data analysis. I participated in ATLAS masterclasses by preparing and giving lectures and instructing the students in the hands-on session.%
}
\cventry{}{Science Slams}{}{}{}{%
Science slams are competetive events in which scientists present their research in a given time frame to a diverse audience in an entertaining way. I participated in over 30 such events with a \href{https://www.youtube.com/watch?v=ZBDvvXhFoZg}{talk about dark matter searches at the Large Hadron Collider}, including the Southern German championship.%
}


% Publications from a BibTeX file without multibib
%  for numerical labels: \renewcommand{\bibliographyitemlabel}{\@biblabel{\arabic{enumiv}}}% CONSIDER MERGING WITH PREAMBLE PART
%  to redefine the heading string ("Publications"): \renewcommand{\refname}{Articles}
\nocite{*}
\bibliographystyle{plain}
\bibliography{publications}                        % 'publications' is the name of a BibTeX file

% Publications from a BibTeX file using the multibib package
%\section{Publications}
%\nocitebook{book1,book2}
%\bibliographystylebook{plain}
%\bibliographybook{publications}                   % 'publications' is the name of a BibTeX file
%\nocitemisc{misc1,misc2,misc3}
%\bibliographystylemisc{plain}
%\bibliographymisc{publications}                   % 'publications' is the name of a BibTeX file

\end{document}