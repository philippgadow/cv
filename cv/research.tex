%-------------------------------------------------------------------------------
%	SECTION TITLE
%-------------------------------------------------------------------------------
\cvsection{Research Experience}


%-------------------------------------------------------------------------------
%	CONTENT
%-------------------------------------------------------------------------------

\cvsubsection{Current Research Activity}

\begin{cventries}
%---------------------------------------------------------
\cventry
  {CMS Experiment, ATLAS Experiment} % Experiment
  {Heavy Flavour Jet Tagging} % Topic
  {U. of Hamburg, CERN, DESY} % Location
  {2021 -- Present} % Date(s)
{
  \begin{cvitems} % Description(s)
    \item {Flavour tagging is a critical component of the ATLAS physics programme, essential for identifying jets originating from \(b\)- and \(c\)-quarks, which is vital for measurements involving Higgs bosons, top quarks, and searches for BSM physics. The use of advanced attention-based machine learning
    techniques using models with transformer encoders resulted in a fourfold improvement in
    background rejection compared to previous methods.}
    \item \textbf{Scope of research:} I have pioneered advanced attention-based machine learning techniques
    using the transformer architecture and deep neural networks for flavour tagging, significantly
    improving background rejection by a factor of four compared to previous methods. In my role
    as sub-group convener for the ATLAS flavour tagging algorithms group, I have led the
    change in the group not only in using groundbreaking methods but also towards modern
    software with excellent documentation, training material and automated testing. I have
    contributed to software design and produced the datasets for training of the algorithms. In my
    two years as sub-group convener, I oversaw the projects of 20 doctoral students qualifying as
    ATLAS authors. I supervised an undergraduate student project on training charged particle
    track-based jets with variable jet radius with a Deep Sets algorithm. Further, I am supervising
    a master's thesis project that enhances the identification of \(c\)-jets with charged \(D^{\ast}\) mesons by
    using an auxiliary training objective to detect "slow pion" tracks from \(D^{\pm \ast}\) decays. I have led
    the publication describing the flavour tagging algorithms used in the major part of the Run 2
    ATLAS physics programme. I quantified the performance of various algorithms for simulated
    data of Run 2 and Run 3 detector configurations.
    \item \textbf{Main collaborators:} Prof. Tim Scanlon (UCL), Dr. Samuel Van Stroud (UCL), Dr. Nicole
    Hartman (TUM), Dr. Daniel Guest (HU Berlin), Dr. Francesco Armando Di Bello (INFN,
    University of Genova), Dr. Andrea Knue (University of Freiburg), Dr. Spyros Argyropoulos (University of Freiburg), Dr. Maximilian Goblirsch-Kolb (CERN), Dr. Markus Elsing (CERN),
    Dr. Michael Kagan (SLAC).
  \end{cvitems}
}
%---------------------------------------------------------
\cventry
  {CMS Experiment} % Experiment
  {Search for non-resonant Higgs pair production} % Topic
  {University of Hamburg} % Location
  {2024 -- Present} % Date(s)
{
  \begin{cvitems} % Description(s)
    \item {Investigation of the non-resonant production of Higgs boson pairs to explore the nature of electroweak symmetry breaking and self-interaction of the Higgs field.}
    \item \textbf{Scope of Research:} Involvement in the analysis of the \textit{bb}\ensuremath{\tau}\ensuremath{\tau} decay channel, focusing on advanced signal extraction techniques, background modeling, and interpretation of results.   
    \item \textbf{Main collaborators:} Prof. Kostas Nikolopoulos (University of Hamburg), Prof. Peter Schleper (University of Hamburg), Dr. Robert Ward.
  \end{cvitems}
}
%---------------------------------------------------------
\cventry
{Phenomenology} % Experiment
{Phenomenology of Extended Higgs Sectors with Top Quarks} % Topic
{DESY Hamburg, CERN} % Location
{2022 -- Present} % Date(s)
{
  \begin{cvitems} % Description(s)
    \item {Models with an extended Higgs sector predicting two or more Higgs bosons can exhibit significant interference effects not only between signal and background but also among the signals themselves.
    These interference effects can be substantial, leading to drastic modifications in predictions,
    resulting in different line-shapes and cross-sections in final states with two, three, or four top quarks. Only a comprehensive analysis investigating the presence of heavy resonances across all top quark multiplicities will be shown to conclusively search for these new particles.}
    \item \textbf{Scope of research:} I investigated interference effects in final states with two, three and four top quarks on parton level in invariant mass distributions of top quarks originating from a scalar. I interpreted predictions with a recast of an LHC four-top-quark search to estimate the expected discovery significance of the different production modes in presence of interference effects.
    \item \textbf{Main collaborators:} Prof. Georg Weiglein (DESY), Dr. Henning Bahl (Heidelberg),
    Dr. Panagiotis Stylianou (DESY), Dr. Krisztian Peters (DESY).
  \end{cvitems}
}

%---------------------------------------------------------
\end{cventries}

\cvsubsection{Previous Research Activity}
\begin{cventries}
%---------------------------------------------------------
\cventry
{ATLAS Experiment} % Experiment
{Heavy Resonance Rearches with Top Quarks} % Topic
{DESY Hamburg, CERN} % Location
{(2020 – 2024)} % Date(s)
{
  \begin{cvitems} % Description(s)
    \item {Searches for spin-1 and spin-0 heavy new bosons in challenging final states with three or four top quarks. This research is motivated by its complementarity to conventional searches in di-top-quark final states and further by the recent discovery of four-top-quark production with a moderate excess.}
    \item \textbf{Scope of research:} I led an analysis team of 17 researchers coordinating the analysis of
    ATLAS data recorded during Run 2. I maintained the analysis software and contributed to fitting studies. Further, I studied the simulation of signal processes. From 2023 – 2024 I led
    an analysis team of 29 researchers, coordinating the combined analysis of ATLAS data taken
    during Run 2 and Run 3. I contributed to analysis strategy design, studies of background
    processes, simulations of signal and background processes and the coordination of distributed
    computing tasks. I have stepped down from the coordination role to avoid conflicts of interest
    with my sub-group convene role in which I should review and advise on the team's research.
    \item \textbf{Main collaborators:} Dr. Krisztian Peters (DESY), Prof. Jianming Qian (University of Michigan), Prof. Reinhard Schwienhorst (Michigan State University), Dr. Binbin Dong (Michigan State University), Dr. Nedaa-Alexandra Asbah (CERN).
  \end{cvitems}
}

%---------------------------------------------------------
\cventry
{ATLAS Experiment} % Experiment
{Searches for Vector-like Quarks} % Topic
{DESY Hamburg, CERN} % Location
{(2020 – 2024)} % Date(s)
{
  \begin{cvitems} % Description(s)
    \item {Vector-like Quarks (VLQs) are hypothetical colour triplet, spin-1/2 fermions whose left- and right-handed components have the same electro-weak transformations. They appear in several models beyond the SM, such as extra dimensions and Composite Higgs models, which address the
    naturalness issue of the Higgs boson mass. While low-mass VLQs are pair-produced via the strong
    interaction, higher masses suppress this process, making electroweak single production significant.}
    \item \textbf{Scope of research:} I contributed to a search for the single production of VLQs decaying to a \(b\)-quark and a Higgs boson, using the full ATLAS Run 2 dataset, employing advanced jet
    substructure techniques and flavour tagging algorithms with studies of the signal process in
    consultation with theorists and simulation of the signal process. I also managed a search for
    single production of VLQs decaying to a \(b\)-quark and a Higgs boson decaying to two photons
    from 2021 to 2023, working together with a single doctoral researcher. I wrote the analysis
    software, simulated the signal process, studied the background prediction based on control
    region data including the estimate of associated uncertainties and performed initial fit studies. I am mentoring the doctoral student who has taken over the leadership of the analysis in 2023. Further, I am guiding the first searches for VLQs using ATLAS Run 3 data in my role as HQT sub-group convener and will be advising the teams and reviewing their results.
    \item \textbf{Main collaborators:} Dr. Marco Montella (Ohio State University), Dr. Krisztian Peters (DESY), Prof. Jahred Adelman (Northern Illinois University), Dr. Elin Bergeås Kuutmann (Uppsala University), Dr. Natascia Vignaroli (Salento University).
  \end{cvitems}
}


%---------------------------------------------------------
\cventry
{ATLAS Experiment} % Experiment
{Lepton Isolation Techniques} % Topic
{CERN} % Location
{2023 -- 2024} % Date(s)
{
\begin{cvitems} % Description(s)
\item {Lepton isolation techniques distinguish high-energy leptons originating from electroweak processes from those produced in semi-leptonic hadron decays, photon conversions, or mis-identified
particles. They play a crucial role in the ATLAS physics programme for Higgs boson
measurements, electroweak processes, and physics searches for phenomena Beyond the Standard Model. New approaches based on algorithms used as well for the identification of heavy flavour jets reduce backgrounds significantly and consequently enhance the sensitivity of multi-lepton final state analyses.}
\item \textbf{Scope of research:} I developed the "Prompt Lepton Isolation Tagger" using a multi-task
transformer neural network to identify leptons from \(W / Z\) boson decays. This involved
creating the necessary software to produce data products for training, generating large,
preprocessed training datasets with balanced classes, and training initial models. Additionally,
I implemented an ONNX-runtime-based version of the algorithm into the central ATLAS
software. I also mentor a student who evaluates the algorithm's performance for muons in
simulated data.
\item \textbf{Main collaborators:} Dr. Frédéric Déliot (Saclay CEA), Dr. Henri Bachacou (Saclay), Prof. Jelena Jovičić (Belgrade IP), Dr. Marco Vanadia (Rome "Tor Vergata"), Dr. Nello Bruscino
(Rome "La Sapienza"), Dr. Francisco Alonso (La Plata University), Dr. Nedaa-Alexandra
Asbah (CERN), Dr. Knut Zoch (Harvard), Dr. Maximilian Goblirsch-Kolb (CERN).
\end{cvitems}
}

%---------------------------------------------------------
\cventry
{CERN EP-R\&D} % Experiment
{Silicon Detector R\&D for Future Collider Facilities} % Topic
{CERN} % Location
{2023 -- 2024} % Date(s)
{
\begin{cvitems} % Description(s)
\item {The future of high-energy physics relies on advancing collider capabilities, with priorities such as an electron-positron Higgs factory for precision measurements and new physics exploration.
Contributing to this goal, I study silicon pixel detector technologies for future collider facilities, evaluating pixel detectors fabricated using a 65 nm CMOS process in collaboration with DESY, as part of CERN EP department's strategic research and development programme on technologies for future experiments and detector technologies.}
\item \textbf{Scope of research:} I have characterised hybrid CLICpix2 sensors in the laboratory, calibrating the threshold settings of individual pixels to ensure uniform response across all pixels of the sensor, and performed a charge calibration with sources and fluorescence measurements with x-rays to obtain the absolute charge created by particle interactions in the active detector medium. I performed similar measurements for the novel H2M ("hybrid-to-monolithic")
monolithic active pixel sensor, including determination of pixel noise. I participated in test-
beam measurements at the CERN SPS facility of the H2M sensor and contribute to test-beam
data analysis.
\item \textbf{Main collaborators:} Dr. Dominik Dannheim (CERN), Dr. Peter Švihra (CERN), Dr. Younes
Otarid (CERN), Dr. Michael Campbell (CERN), Dr. Simon Spannagel (DESY), Dr. Lennart
Huth (DESY), Dr. Finn Feindt (DESY).
\end{cvitems}
}

  %---------------------------------------------------------
  \cventry
    {ATLAS Experiment} % Experiment
    {Searches for Dark Matter at Colliders} % Topic
    {Max Planck Institute for Physics} % Location
    {2016 -- 2020} % Date(s)
    {
      \begin{cvitems} % Description(s)
        \item {The Standard Model does not account for over 80\% of the universe’s gravitating matter – dark matter. Searches at the Large Hadron Collider aim to probe signatures of yet unobserved dark matter particles by detecting their recoil on visible particles in signatures with missing transverse momentum. I contributed to three such searches using heavy bosons: hadronically decaying vector bosons, Higgs bosons decaying to \(b\)-quarks, and hypothetical dark Higgs bosons decaying to pairs of vector bosons, all of which pioneered new analysis techniques and constrained models describing dark matter production.}
        \item \textbf{Scope of research:} My contributions to dark matter searches have been extensive. In the search
        for dark matter produced in association with hadronically decaying vector bosons, I served as
        one of two central analysers and was responsible for developing and maintaining the analysis
        software, managing distributed computing, and conducting the statistical analysis. I developed
        a method for estimating the multi-jet background and applied it successfully. Further, I
        facilitated the incorporation of the results into a summary publication, their reinterpretation
        within an extended Higgs sector model, and their use in combined searches for invisible Higgs
        boson decays. For dark matter searches involving Higgs bosons decaying to \(b\)-quarks, I was a
        central analyser for the partial Run 2 data (2015--2017) and contributed to the full Run 2 result.
        My responsibilities included managing the analysis software, producing fit inputs, conducting
        statistical analysis, and commissioning a new object-based missing transverse momentum
        significance. I also estimated the multi-jet background and maintained the derivation datasets,
        as well as optimising selection requirements based on the new significance measure. In the
        search for dark Higgs bosons decaying to vector boson pairs, I was the central analyser,
        handling all critical aspects of this novel analysis. My contributions included simulating the
        signal process, designing the analysis strategy, developing the analysis software, managing
        distributed computing, and conducting the statistical analysis.
        \item \textbf{Main collaborators:} Dr. Sandra Kortner (MPP), Dr. Patrick Rieck (NYU), Dr. Krisztian Peters (DESY), Dr. Xuanhong Luo (Stockholm University), Dr. Katharina Behr (DESY), Prof. Oleg Brandt (Cambridge), Prof. Shi-Chieh Hsu (Washington University), Prof. Frank Filthaut
        (Nijmegen), Dr. Ruth Pöttgen (Lund University), Prof. Daniel Whiteson (UC Irvine), Prof. Lauren Tompkins (Stanford), Dr. Spyros Argyropoulos (University of Freiburg), Dr. Dan Guest (HU Berlin).
      \end{cvitems}
    }

%---------------------------------------------------------
  \cventry
    {ATLAS Experiment} % Experiment
    {Reinterpretation of New Physics Searches with RECAST} % Topic
    {Max Planck Institute for Physics, DESY Hamburg} % Location
    {2018 -- 2022} % Date(s)
    {
      \begin{cvitems} % Description(s)
        \item {Searches for new physics often involve significant investments of time and resources, making reinterpretation frameworks like RECAST essential for efficiently testing alternative signal hypotheses.}
        \item \textbf{Scope of research:} I coordinated and co-edited the first ATLAS public result using the reinterpretation RECAST framework, which allows for the semi-automated reuse of background estimates,
        systematic uncertainties, and observed data to test alternative signal hypotheses, setting
        constraints on a dark matter model with a dark Higgs boson decaying to \(b\)-quarks.
        Additionally, I contributed to the study of an Active Learning approach to extend parameter
        scans efficiently, reducing the computational effort needed while expanding the dimensionality
        of the parameter space examined. For the latter, I implemented a fast parametrisation of the
        analysis and automated the simulation of the signal process.
        \item \textbf{Main collaborators:} Prof. Lukas Heinrich (TUM), Prof. Kyle Cranmer (Wisconsin),
        Dr. Patrick Rieck (NYU), Prof. Shih-Chieh Hsu (Washington University).
      \end{cvitems}
    }

%---------------------------------------------------------
  \cventry
    {ATLAS Experiment} % Experiment
    {Muon Trigger Improvements for Run 2 and Run 4} % Topic
    {Max Planck Institute for Physics} % Location
    {2015 -- 2017} % Date(s)
  {
    \begin{cvitems} % Description(s)
      \item {Highly selective first muon level triggers are essential to exploit the full physics potential of the ATLAS experiment.}
      \item \textbf{Scope of research:} I optimised the first-level ATLAS muon trigger for the Run 2 data-taking
      campaign by refining the trigger coincidence logic using the tag-and-probe method with muon
      pairs from the \(J/\psi\) resonance, resulting in reduced trigger rates in the forward region while
      maintaining high efficiency for 2018 data-taking. For the ATLAS Run 4 muon trigger concept
      study, I estimated trigger rates incorporating precision tracking information from the
      monitored drift tube chambers and outlined a new pattern recognition strategy, which has been
      adopted as the baseline for the Phase-II muon trigger upgrade.
      \item \textbf{Main collaborators:}Dr. Oliver Kortner (MPP), Dr. Sandra Kortner (MPP), Prof. Yasuyuki Horii (Nagoya University), Prof. Junpei Maeda (Kobe University).
    \end{cvitems}
  }
%---------------------------------------------------------
\end{cventries}
